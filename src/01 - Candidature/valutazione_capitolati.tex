\documentclass{TWDocumentFull}

\title{Valutazione capitolati gruppo 5}

\editor{Carraro Agnese}
\editor{Marcon Giulia}
\editor{Pistori Gaia}
\editor{Vasquez Manuel Felipe}

\reviewer{Monetti Luca}
\reviewer{Piola Andrea}
\reviewer{Dal Bianco Riccardo}
\classification{Esterno}
\version{1.0}

\begin{document}
    \frontmatter
    
    \section*{C1: Artificial QI}
    {\textbf{Proponente}: Zucchetti}
    \paragraph{Descrizione:\\}
    {Viene richiesto di creare un software per il confronto e la valutazione di vari sistemi che utilizzano i Large Language Models. È necessario stabilire un metodo di confronto tra i sistemi basato sulle loro risposte, memorizzare una lista di domande e di risposte attese e fornire un supporto per la visualizzazione dei risultati ottenuti.}
    \paragraph{Criticità:}
    \begin{itemize}
        \item Difficoltà nel stabilire un metodo di confronto affidabile per testi in linguaggio naturale senza l'utilizzo degli LLM stessi.
    \end{itemize}
    \paragraph{Aspetti positivi:}
    \begin{itemize}
        \item Grande supporto da parte dell'azienda, sia nella formazione che negli strumenti offerti.
        \item Progetto interessante in quanto offre una prospettiva diversa sull'Intelligenza Artificiale mettendone in discussione l'affidabilità delle risposte.
    \end{itemize}


    \section*{C2: Vimar GENIALE}
    {\textbf{Proponente}: VIMAR}
    \paragraph{Descrizione:\\}
    {Viene richiesto di creare una web-app responsive che supporti il lavoro degli installatori fornendo informazioni e indicazioni riguardo ai prodotti dell'azienda tramite una chat. \\ La chat permette di porre domande sia tramite template guidati che con testo libero e la risposta viene generata da un LLM addestrato sui dati disponibili nel sito aziendale.
}
    \paragraph{Criticità:}
    \begin{itemize}
        \item Richiede l'utilizzo di diverse tecnologie, molte delle quali non conosciute da alcuni membri del gruppo.
    \end{itemize}
    \paragraph{Aspetti positivi:}
    \begin{itemize}
        \item Capitolato chiaro ed esaustivo che offre anche una bozza del risultato finale.
    \end{itemize}

    \newpage\section*{C3: Automatizzare le routine digitali tramite l'intelligenza generativa}
    {\textbf{Proponente}: Var Group S.p.A.}
    \paragraph{Descrizione:\\}
    {Viene richiesto lo sviluppo di un “servizio ad agenti” che permetta, tramite un client con un'interfaccia drag \& drop, di disegnare, con l'assistenza di un IA, una serie di azioni automatizzate che interagiscono con software locali. Questo servizio permetterebbe di ridurre il carico di lavoro nelle azioni quotidiane, le quali spesso sono molto ripetitive e possono portare a perdite di tempo considerevoli nell'arco di una giornata.}
    \paragraph{Criticità:}
    \begin{itemize}
        \item Creazione di un'applicazione nativa che richiede un dispendio maggiore di risorse.
        \item Il gruppo per la maggior parte non possiede familiarità con le API della suite aziendale di Microsoft.
    \end{itemize}
    \paragraph{Aspetti positivi:}
    \begin{itemize}
        \item Progetto con ampio margine creativo per la decisione delle azioni da automatizzare.
        \item Progetto impiegabile anche in contesti esterni all'azienda proponente.
    \end{itemize}
    \paragraph{Contatti:}
    \begin{itemize}
        \item Scambio di email per chiarimenti.
    \end{itemize}

    \section*{C4: NearYou - Smart custom advertising platform}
    {\textbf{Proponente}: Sync Lab}
    \paragraph{Descrizione:\\}
    {Viene richiesto di sviluppare uno strumento di advertising personalizzato tramite i LLM, in modo da creare e proporre agli utenti degli annunci che possano essere di loro interesse. Gli annunci vengono visualizzati dall'utente tramite una web-app presente su un display nel loro veicolo e scelti in base a dei dati simulati come le loro informazioni personali, il loro stato fisico e la loro posizione attuale. }
    \paragraph{Criticità:}
    \begin{itemize}
        \item Difficoltà nel garantire che gli annunci generati siano coerenti con l'identità dell'azienda.
        \item Difficoltà nella gestione dell'invio dei dati in tempo reale e nella scelta di una frequenza adeguata.
    \end{itemize}
    \paragraph{Aspetti positivi:}
    \begin{itemize}
        \item Capitolato chiaro ed esaustivo.
        \item Grande disponibilità da parte dell'azienda di supporto al team di sviluppo.
    \end{itemize}

    \section*{C5: 3Dataviz}
    {\textbf{Proponente}: Sanmarco Informatica}
    \paragraph{Descrizione:\\}
    {Viene richiesto di realizzare un'interfaccia web per la visualizzazione in forma tridimensionale di dati (ad esempio meteo, voli, stock, … ) tramite barre verticali (istogramma 3D) attraverso l'implementazione di funzionalità per rendere più facile la lettura e la visualizzazione dei dati di origine tramite un modello navigabile ed interattivo.}
    \paragraph{Criticità:}
    \begin{itemize}
        \item Attualmente non esiste una tecnologia che permetta di rappresentare grafici in 3D, richiedendo quindi di adattare le librerie presenti.
    \end{itemize}
    \paragraph{Aspetti positivi:}
    \begin{itemize}
        \item Grande supporto da parte dell'azienda tramite l'affiancamento di personale esperto.
    \end{itemize}
    \paragraph{Contatti:}
    \begin{itemize}
        \item Scambio di email per chiarimenti.
    \end{itemize}

    \section*{C6: Sistema di Gestione di un Magazzino Distribuito}
    {\textbf{Proponente}: M31}
    \paragraph{Descrizione:\\}
    {Viene richiesto di creare un sistema distribuito e scalabile per la gestione dell'inventario di una rete di magazzini. L'obiettivo è garantire un flusso continuo di materiali e prodotti tra magazzini lontani geograficamente, minimizzando il rischio di esaurimento scorte e migliorando i tempi di risposta alle variazioni di domanda. \\L'architettura deve essere basata su microservizi per garantire l'autonomia dei magazzini. Ogni magazzino dispone di servizi per la gestione locale delle operazioni di inventario, che si sincronizza con una copia remota su cloud.
}
    \paragraph{Criticità:}
    \begin{itemize}
        \item Difficoltà nella sincronizzazione di un sistema distribuito altamente performante.
        \item Difficoltà di ricreare un ambiente di test verosimile durante la fase di sviluppo.
        \item Requisiti tecnici complessi e con aspettative molto alte.
    \end{itemize}
    \paragraph{Aspetti positivi:}
    \begin{itemize}
        \item Capitolato chiaro ed esaustivo.
        \item Possibilità di lavorare con un'architettura distribuita e quindi difficilmente replicabile in un progetto personale.
    \end{itemize}

    \section*{C7: Assistente Virtuale}
    {\textbf{Proponente}: Ergon}
    \paragraph{Descrizione:\\}
    {Viene richiesta la creazione di un assistente virtuale per le aziende che operano nella vendita di prodotti con cataloghi estesi. Questo assistente sfrutterà le potenzialità degli LLM per analizzare e interpretare i dati aziendali, migliorando l'interazione cliente-azienda. \\Il sistema permette di semplificare l'accesso alle informazioni dei prodotti senza la necessità di intervento degli specialisti. Attraverso un chatbot gli utenti potranno ricevere risposte personalizzate alle loro domande sui prodotti disponibili.}
    \paragraph{Criticità:}
    \begin{itemize}
        \item Difficoltà nel preprocessing dei dati provenienti dai vari documenti aziendali.
        \item Difficoltà nell'uso di un database vettoriale in quanto strutturalmente diverso da quelli conosciuti dalla maggior parte del team. 
    \end{itemize}
    \paragraph{Aspetti positivi:}
    \begin{itemize}
        \item Grande disponibilità da parte dell'azienda nel fornire chiarimenti ad eventuali domande.
        \item Chiarezza nella presentazione della proposta di capitolato.
    \end{itemize}
    \paragraph{Contatti:}
    \begin{itemize}
        \item Scambio di email per chiarimenti  e meeting online in data 28 Ottobre 2024.
    \end{itemize}

    \section*{C8: Requirement Tracker - Plug-in VS Code}
    {\textbf{Proponente}: Bluewind}
    \paragraph{Descrizione:\\}
    {In ambienti di sviluppo complessi, come la programmazione embedded, la tracciabilità e la definizione dei requisiti diventa fondamentale. Viene richiesto lo sviluppo di un'estensione per VSCode che permetta agli sviluppatori di caricare una serie di requisiti e, tramite l'utilizzo dell'intelligenza artificiale, verificare la loro corretta implementazione all'interno del codice.}
    \paragraph{Criticità:}
    \begin{itemize}
        \item Contesto dei microprocessori e della programmazione embedded sconosciuto al gruppo.
    \end{itemize}
    \paragraph{Aspetti positivi:}
    \begin{itemize}
        \item Buona disponibilità nell'offrire chiarimenti e spiegazioni.
        \item Ambiente d'uso differente da tutti gli altri contesti.
        \item Non viene richiesta la creazione di un applicativo ma di una estensione per l'ambiente di sviluppo.
    \end{itemize}
    \paragraph{Contatti:}
    \begin{itemize}
        \item Scambio di email per chiarimenti  e meeting online in data 29 Ottobre 2024.
    \end{itemize}

    \section*{C9: BuddyBot}
    {\textbf{Proponente}: azzurrodigitale}
    \paragraph{Descrizione:\\}
    {La proposta riguarda lo sviluppo di un chatbot: un assistente digitale capace di ottenere informazioni tramite le API dei vari strumenti aziendali al fine di fornire agli sviluppatori una risposta rapida e precisa alle loro domande relative ai progetti aziendali.\\Questo permetterebbe di ridurre l'inefficienza causata dalla dispersione delle informazioni nelle varie piattaforme fornendo un singolo punto di accesso a queste ultime.}
    \paragraph{Criticità:}
    \begin{itemize}
        \item Richiede l'utilizzo di diverse API esterne per ottenere le informazioni richieste.
    \end{itemize}
    \paragraph{Aspetti positivi:}
    \begin{itemize}
        \item Presentazione del capitolato molto chiara.
        \item Ambito applicativo vicino agli interessi dei membri del gruppo.
    \end{itemize}

\end{document}