\documentclass{TWDocumentNoVersion}
\usepackage{float}
\title{Candidatura gruppo 5}

\editor{Carraro Agnese}
\editor{Marcon Giulia}
\editor{Pistori Gaia}
\editor{Vasquez Manuel Felipe}
\usepackage{float}
\usepackage{hyperref}
\usepackage{amsmath}

\reviewer{Monetti Luca}
\reviewer{Piola Andrea}
\reviewer{Dal Bianco Riccardo}
\classification{Esterno}
\version{1.0}

\begin{document}
    \frontmatter

    \section*{Partecipanti}
    \begin{itemize}
        \item {Agnese Carraro (2068226)}
        \item{Riccardo Dal Bianco (2042385)}
        \item{Giulia Marcon (2075541)}
        \item{Luca Monetti (2069429)}
        \item{Andrea Piola (2068241)}
        \item{Gaia Pistori (2075531)}
        \item{Manuel Felipe Vasquez (2076425)}
    \end{itemize}

    \section*{Scelta Capitolato}
    {Dopo un'attenta valutazione delle proposte delle aziende, durante la quale il gruppo ha analizzato i contenuti, i requisiti, i punti di forza e le criticità di ognuna di esse, si è giunti alla conclusione di scegliere il capitolato C8: \textit{“\textbf{Requirement Tracker - Plug-in VSCode}”} dell'azienda BlueWind. $\\$ Abbiamo avuto modo di approfondire i requisiti del loro capitolato inizialmente con uno scambio di mail e successivamente con un incontro online il 29 Ottobre 2024.}

    \section*{Motivazioni della scelta}
    {Le motivazioni alla base di tale scelta sono le seguenti:}
    \begin{itemize}
        \item {Il progetto si contraddistingue dagli altri capitolati sia per la tipologia di prodotto, ovvero un plug-in per l'IDE VSCode, sia perché pensato come supporto alla programmazione embedded.}
        \item{L'azienda si è resa molto disponibile e paziente nel fornire chiarimenti su un ampio spettro di domande, con esempi e riferimenti reali.}
        \item{L'azienda si impegna a fornire un caso di studio realistico per supportarci nella creazione del PoC.}
        \item{L'azienda ha proposto un piano di lavoro realistico, in linea con quello da noi preventivato. È prevista una fase iniziale di affiancamento con il team aziendale per acquisire familiarità con i linguaggi e le tecnologie proposte. Successivamente verrà svolta l'analisi dei requisiti e la scelta condivisa delle funzionalità minime da implementare.}
        \item{L'azienda considera inoltre un possibile utilizzo reale dell'applicativo, prevedendone l'impiego in progetti proprietari dei loro clienti. Per questo motivo privilegiano l'utilizzo di Ollama ( Self-Hosted ) rispetto ad alternative in cloud come OpenAI}
    \end{itemize}

    \section*{Valutazione dei capitolati}
    \begin{itemize}
        \item \href{https://tech-wave-swe.github.io/Docs}{https://tech-wave-swe.github.io/Docs}
    \end{itemize}

    \newpage\section*{Dichiarazione impegni}
    {Ci impegniamo a completare lo sviluppo del progetto entro la data 21 marzo 2025. $\\$ Ciascuno membro del team si impegna a lavorare 90 ore produttive per l'intera durata del progetto. $\\$ Si presume che durante il progetto durerà circa 18 settimane produttive, con intensità variabile:}
    \begin{itemize}
        \item{5 settimane con intensità bassa (periodo della sessione: dal 20/01 al 23/02) con un impegno medio individuale di 3 ore produttive a settimana per componente,}
        \item{9 settimane con intensità media (periodo delle lezioni: dal 4/11 al 21/12 e dal 7/01 al 19/01) con un impegno medio individuale di 5 ore produttive a settimana per componente,}
        \item{4 settimane con intensità alta (ultimo mese: dal 24/02 al 21/03) con un impegno medio individuale di 8 ore produttive a settimana per componente,}
        \item{2 settimane di pausa per le vacanze di Natale (dal 22/12 al 6/01).}
    \end{itemize}

    {Il preventivo della spesa totale ammonta a 12.800€ e deriva dalla seguente previsione di distribuzione delle ore di lavoro nei vari ruoli:}

    \renewcommand{\arraystretch}{1.5}

    \begin{table}[H]
        \begin{tabularx}{\textwidth}{|X|X|X|X|} \hline
            \rowcolor{twlightblue}
             \LabelText{Ruolo}  & \LabelText{Costo} & \LabelText{Ore previste} & \LabelText{Totale} \\ \hline
             Responsabile   & 30€ & 75  & 2250€\\ \hline
             Amministratore & 20€ & 85  & 1700€\\ \hline
             Analista       & 25€ & 90  & 2250€\\ \hline
             Progettista    & 25€ & 90  & 2250€\\ \hline
             Programmatore  & 15€ & 170 & 2550€\\ \hline
             Verificatore   & 15€ & 120 & 1800€\\ \hline
             % \rowcolor{twlightblue}
             \multicolumn{3}{|l|}{\LabelText{Totale}}& 12800€\\ \hline
        \end{tabularx}
    \end{table}

    \begin{itemize}
        \item{\textbf{Responsabile}: questo ruolo è necessario per l'organizzazione del team. Si occupa inoltre della comunicazione con l'esterno. Stimiamo quindi che sia necessario un impiego settimanale medio di circa 4 ore per tutta la durata del progetto.}
        \item{\textbf{Amministratore}: questo ruolo richiede un impegno maggiore nelle prime settimane durante le quali dovrà configurare l'ambiente di sviluppo del gruppo. Sarà sempre necessario il suo intervento per attuare modifiche in caso di necessità.}
        \item{\textbf{Analista} e \textbf{progettista}: questi ruoli hanno una richiesta di lavoro equivalente in quanto si concentreranno all'inizio con la creazione del PoC e poi all'avvio della progettazione dell'MVP. Durante questi periodi queste figure lavoreranno a stretto contatto alternando fasi di analisi e progettazione.}
        \item{\textbf{Programmatore}: secondo le nostre previsioni questo ruolo costituirà una delle parti più onerose, a livello di tempo. Stimiamo che, in media, per ogni ora dedicata alla progettazione di una funzionalità siano necessarie circa 2 ore di programmazione per la sua implementazione.}
        \item{\textbf{Verificatore}: questo ruolo è in parte legato a quello del programmatore. Stimiamo che per ogni funzionalità implementata sia necessario dedicare almeno la metà del tempo necessaria al suo sviluppo alla fase di test. Sono inoltre previste delle ore aggiuntive dedicate alla verifica di tutti gli altri artefatti prodotti.}
    \end{itemize}

    \section*{Calendario di massima del progetto}
        \subsection*{Prima Fase: Novembre - Dicembre}
        {In questa prima fase, come concordato con l'azienda durante il meeting, il team si impegnerà in una prima “iterazione” di:}
            \begin{itemize}
                \item Formazione sulle nuove tecnologie
                \item Creazione di un PoC
                \item Analisi dei Requisiti
            \end{itemize}
        \subsection*{Seconda Fase: Gennaio - Marzo}
        {In questa seconda fase il team si concentrerà sulla pianificazione e sullo sviluppo dell'MVP ( Minimum Viable Product ) per rispondere alle richieste dell'azienda. L'obiettivo sarà garantire un prodotto funzionale allineato con le richieste individuate in fase di Analisi dei Requisiti.}

    \section*{Rischi attesi e mitigazioni}
       \begin{table}[H]
        \centering
        \begin{tabularx}{\textwidth}{|X|X|}
            \hline
            \rowcolor{twlightblue}
            \LabelText{Rischio Individuato} & \LabelText{Possibile Mitigazione} \\
            \hline
            \textbf{Avanzamento lento:}
            specialmente nel primo periodo di formazione e di creazione del PoC considerando la richiesta di tecnologie non conosciute da alcuni membri del gruppo.
            &
            Questo rischio può essere mitigato attraverso la pianificazione di incontri frequenti con l'azienda BlueWind che si è resa disponibile a fornire formazione e supporto. Inoltre, vogliamo scegliere un modello di sviluppo incrementale che si integri con il modello AGILE scrum dell'azienda.\\ \hline
            \textbf{Ritardi dovuti a problemi individuali:}
            Un elemento del gruppo potrebbe non essere in grado di completare il lavoro pianificato entro la data prevista creando rallentamenti nel flusso di lavoro.
            &
            Questo rischio può essere mitigato attraverso la comunicazione costante tra i vari membri del gruppo. Questo ci permetterà di individuare difficoltà o ritardi prima che essi possano trasformarsi in problemi.\\
            \hline
            \textbf{Mancanza di conoscenze tecniche:}
            Alcuni membri del gruppo potrebbero non aver familiarità con tutte le tecnologie utilizzate durante lo sviluppo o la gestione del progetto.
            &
            Questo rischio può essere mitigato attraverso una stretta collaborazione tra i membri del team e l'azienda proponente mirando alla creazione di un ambiente di collaborazione e condivisione delle conoscenze acquisite.\\
            \hline
        \end{tabularx}
    \end{table}

\end{document}
\end{document}
