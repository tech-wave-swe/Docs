\documentclass{TWReport}

\title{Verbale 16 ottobre 2024}

\author{Luca Monetti}
\participant[present]{Carraro Agnese}
\participant[present]{Dal Bianco Riccardo}
\participant[present]{Marcon Giulia}
\participant[present]{Monetti Luca}
\participant[present]{Piola Andrea}
\participant[present]{Pistori Gaia}
\participant[present]{Vasquez Manuel Felipe}

\editor{Luca Monetti}

\reviewer{Vasquez Manuel Felipe}
\classification{Interno}
\version{1.0}

\begin{document}

\frontmatter

\showPartecipants

\section*{Ordine del giorno}
\begin{itemize}
    \item Decisione nome del gruppo e relativo logo
    \item Creazione email del gruppo di lavoro
    \item Discussione preferenze capitolati
    \item Primo contatto con aziende per chiarimenti sui capitolati
\end{itemize}

\section*{Resoconto}
\begin{itemize}
    \item Sono state analizzate diverse opzioni per il nome e, successivamente per il logo, del team. La decisione finale è stata presa tramite votazione.
    \item In seguito è stato predisposto un indirizzo di posta elettronica comune, uno spazio condiviso per i documenti comuni ( File di documentazione, verbali … ) ed i canali di comunicazione di base.
    \item Per ogni capitolato sono stati analizzati interessi e perplessità dei vari membri. Da questo momento di discussione sono state estrapolate alcune domande riguardanti i punti più ambigui dei vari capitolati.
    \item Sono state infine contattate le aziende corrispondenti al fine di ottenere chiarimenti sul capitolato proposto e verificare la comprensione delle funzionalità richieste.

\end{itemize}

\section*{Conclusioni raggiunte}
\begin{itemize}
    \item Si è deciso di adottare il nome “TechWave” ed il logo presente in prima pagina.
    \item Si è decisa una prima bozza della graduatoria dei capitolati ordinandoli per interesse.
    \item Si è deciso di analizzare ulteriormente la proposta delle aziende VARGROUP e BlueWind.
\end{itemize}

\end{document}
