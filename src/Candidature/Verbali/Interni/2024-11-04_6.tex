\documentclass{TWReport}

\title{Verbale 6: 4 Novembre 2024}

\author{Piola Andrea}
\participant[present]{Carraro Agnese}
\participant[present]{Dal Bianco Riccardo}
\participant[present]{Marcon Giulia}
\participant[present]{Monetti Luca}
\participant[present]{Piola Andrea}
\participant[present]{Pistori Gaia}
\participant[present]{Vasquez Manuel Felipe}

\editor{Piola Andrea}

\reviewer{Pistori Gaia}
\duration{2h}
\classification{Interno}
\version{1.0}

\begin{document}

\frontmatter

\begin{table}[h]
  \centering
  \renewcommand{\arraystretch}{1.5}
    \rowcolors{8}{twlightblue}{twlightblue}%
        \begin{tabularx}{\textwidth}{|>{\centering\arraybackslash}X|>{\centering\arraybackslash}X|>{\centering\arraybackslash}X|>{\centering\arraybackslash}X|>{\centering\arraybackslash}X|>{\centering\arraybackslash}X|}
        \hline
        \textbf{Data Modifica} & \textbf{Versione} & \textbf{Annotazione} & \textbf{Autore Modifica} & \textbf{Data Approvazione} & \textbf{Autore Approvazione} \\
        \hline
        {04/11/2024} & {1.0} & {prima stesura del documento} & {Andrea Piola} & {05/11/2024} & {Pistori Gaia} \\
        \hline
    \end{tabularx}
\end{table}
\newpage

\showPartecipants

\section*{Ordine del giorno}
\begin{itemize}
    \item Visione delle problematiche legate alla candidatura.
    \item Discussione sulla soluzione alle problematiche riscontrate.
    \item Correzione della precedente candidatura.
\end{itemize}

\section*{Resoconto della riunione}
\begin{itemize}
    \item Il team ha preso in considerazione le problematiche legate alla candidatura ed ha cercato delle soluzioni utili alle stesse.
    \item Revisione della candidatura con considerazioni sulle modifiche da effettuare.
    \item Valutazione di aggiunte del registro delle modifiche e della durata delle riunioni nei verbali.
    \item Gestione della repository di Github.
\end{itemize}

\section*{Decisioni Prese}
\begin{enumerate}
    \item Modifica della candidatura con aggiunta delle voci mancanti.
    \item Correzione dei verbali con le nuove aggiunte.
    \item Creazione di pagina web per migliorare la fruibilità della repository.
\end{enumerate}

\end{document}
